% Created 2020-09-06 Sun 23:10
% Intended LaTeX compiler: xelatex
\documentclass[10pt]{beamer}
\usepackage{fontspec}
\usepackage{graphicx}
\usepackage{grffile}
\usepackage{array}
\usepackage{tabularx}
\usepackage{booktabs}
\usepackage{multirow}
\usepackage{siunitx}
\usepackage{wrapfig}
\usepackage{rotating}
\usepackage[normalem]{ulem}
\usepackage{amsmath}
\usepackage{mathrsfs}
\usepackage{textcomp}
\usepackage{amssymb}
\usepackage{capt-of}
\usepackage[dvipsnames]{xcolor}
\usepackage[colorlinks=true, linkcolor=Blue, citecolor=BrickRed, urlcolor=PineGreen]{hyperref}
\usepackage{indentfirst}
\usepackage{minted}
\usepackage{adjustbox}
\usepackage[backend=bibtex,sorting=ydnt,style=authoryear]{biblatex}
\AtBeginBibliography{\footnotesize}
\addbibresource{~/Dropbox/Notes/Research/papers.bib}
\usepackage{xeCJK}
\usefonttheme{professionalfonts}
\usetheme{metropolis}
\usecolortheme{}
\usefonttheme{}
\useinnertheme{}
\useoutertheme{}
\author{Hieu Phay}
\date{2020-09-06}
\title{Dynamic optimization: the Ramsey-Cass-Koopman model}

\setbeamercolor{alerted text}{fg=red!70!black}
\hypersetup{
 pdfauthor={Hieu Phay},
 pdftitle={Dynamic optimization: the Ramsey-Cass-Koopman model},
 pdfkeywords={},
 pdfsubject={},
 pdfcreator={Emacs 28.0.50 (Org mode 9.4)}, 
 pdflang={English}}
\begin{document}

\maketitle

\section{Introduction}
\label{sec:orgdb10c95}
\begin{frame}[label={sec:orge74cf71}]{The Ramsey-Cass-Koopmans (RCK) model}
\begin{itemize}
\item \textcite{Ramsey1928} introduced a model of optimal savings. The model was not well-received at that time because it was too mathematically demanding.
\item Almost thirty years later, the model was getting tractions when \textcite{RePEc:cwl:cwldpp:163} and \textcite{Cass1965} formalized and made extensions to the original model.
\end{itemize}

This model is commonly refered to as the \alert{Ramsey-Cass-Koopmans model of optimal growth}.
\end{frame}

\begin{frame}[label={sec:orgb344fd7}]{Why this model}
\begin{itemize}
\item The RCK model is iconic in neoclassical growth theories, and provides a natural benchmark case.
\item The RCK model examine dynamics optimizations of over an infinite horizon, which makes it much easier to approach from the optimal control PoV
\item The RCK model endogenize consumption, which makes it one of the first micro-founded models.
\end{itemize}
\end{frame}

\section{Building blocks of the model}
\label{sec:org0ac65d9}
\begin{frame}[label={sec:orga703f67}]{Production function}
There is a large number of identical firms, producing a common production . Each has access to the production function:

$$
Y = \Phi(K, L)
$$

Which is \alert{homogeneous} with both \(K\) and \(L\).

We use the lowercase form \(y\) and \(k\) to denote output per labour and capital per labour:

$$
y = \phi(k)
$$

where \(\phi^{\prime}(\cdot) > 0\) and \(\phi^{\prime\prime}(\cdot) < 0\), \(\lim_{k \to 0} \phi^{'}(k) = \infty\) and \(\lim_{k \to \infty} \phi^{'}(k) = 0\)
\end{frame}

\begin{frame}[label={sec:org53c2c76}]{Firms}
Firms face the constraints in capital:

$$
\dot{K} = I - \delta K = Y - C - \delta K
$$

Expanding \(\dot{K}\) into \(\frac{d(kL)}{dt}\), then use the product rule and divide \(L\) from both sides in the above equation give us:

$$
\dot{k} = \phi(k) - c - (n + \delta)k
$$

Also consumptions should be less than income from the same time period: \(0 \leq c(t) \leq \phi[k(t)]\)
\end{frame}

\begin{frame}[allowframebreaks]{Utility function}
The household's utility function takes the form

$$
U = \int_{t=0}^{\infty} e^{-\rho t}U(c(t))L(t)dt
$$

where
\begin{itemize}
\item \(\rho\) is the \alert{discount rate}, the greater is \(\rho\), the less the household values futures consumption relative to current consumption.
\item \(U(c(t))\) is the \alert{instantatneous utility function},
\end{itemize}

$$
\int_0^{\infty}U(c)L(t)e^{-\rho t} dt = \int_0^{\infty} U(c) L_{0} e^{nt} e^-{\rho t} dt = L_0 \int_{0}^{\infty} U(c)e^{-(\rho-n)}dt
$$


\framebreak

\(U(c(t))\) usually take the \alert{constant-relative-risk-aversion} form (\cite{arrow1965aspects,Pratt1964}):

$$
U(c(t)) = \frac{c(t)^{1-\theta}}{1-\theta}, \quad \theta  0, \quad \rho - n - (1 - \theta) > 0
$$

where:
\begin{itemize}
\item \(\theta\) represents the household's willingness to shift consumption between different periods: elasticity of subtitution between consumption at any two point in time is \(\frac{1}{\theta^{2}}\)
\item The relative risk aversion \(RRA = -\frac{cU^{''}(c)}{U^{'}(c)}= \theta\)   is constant
\item \(\rho - n - (1 - \theta) > 0\) ensures that lifetime utilities do not diverge.
\end{itemize}

In special case of \(\theta \to 1\), the instantaneous utility function simplifies to \(ln(c)\).
\end{frame}

\section{Model Analysis}
\label{sec:org766ec81}
\begin{frame}[label={sec:org71983be}]{The optimization problem}
\begin{align*}
\text{Maximize} \quad & \int_0^{\infty} U(c)e^{-rt} dt \\
\text{s.t.} \quad     & \dot{k} = \phi(k) -c - (n + \delta)k \\
                  & k(0) = k_{0} \\
\text{and} \quad      & 0 \leq c(t) \leq \phi[k(t)]
\end{align*}
\end{frame}

\begin{frame}[label={sec:org2de348a}]{The optimal control problem}
For the Hamiltonian:

$$
H = U(c)e^{rt} + \lambda\left[\phi(k) - c - (n + \delta) k \right]
$$

The equations-of-motion conditions are written as:

\begin{equation}
  \dot{\lambda} = - \frac{\partial H}{\partial k} \Leftrightarrow \dot{\lambda} = -\lambda[\theta^{'}(k) - (n + \delta)]
\end{equation}

\begin{equation}
  \dot{k} = \theta(k) - c - (n + \delta) k
\end{equation}


Since \(H\) is convex and \(c\) is unrestricted, we can accordingly find the maximum of \(H\) by setting:

\begin{equation}
\frac{\partial H}{\partial c} = U^{\prime}(c) e^{-rt} - \lambda = 0 \Leftrightarrow U^{\prime}(c) = \lambda e^{rt}
\end{equation}
\end{frame}

\begin{frame}[label={sec:orgeb508ef}]{The optimal control problem}
Examination on convexity of \(H\) (Fig. 9.2 \cite{chiang2000elements}):

\begin{center}
\includegraphics[width=.9\linewidth]{/home/hieuphay/Dropbox/Notes/.attach/c2/6256cf-a3c2-41ac-8ef8-ae6b8fcbae72/_20200904_060843screenshot.png}
\end{center}
\end{frame}

\begin{frame}[allowframebreaks]{The Current-Value Hamiltonian}
In economics, the integrand function \(F\) often contains a \alert{discount factor} \(e^{\rho t}\):

$$
F(t,y,u) = G(t,y,u)e^{\rho t}
$$

We define a new multiplier \(m\) such that:

$$
m = \lambda e^{\rho t}
$$

then \(H_c \equiv He^{\rho t} = G(t, y, u) + mf(t,y,u)\)

\(G\) is called the \alert{Instanteous Utility Function}.

\framebreak

The new conditions can be rearranged as

\begin{enumerate}
\item \(\max_{u} H_c \quad \text{for all } t \in [0, T]\)
\item \(\frac{\partial H_c}{\partial y} = - \dot{m} + \rho m\)
\item \(\frac{\partial H}{\partial u} = 0\)
\item And a transversality condition
\end{enumerate}
\end{frame}

\begin{frame}[label={sec:org66d6089}]{The optimal control problem (rephrased)}
For the current-value Hamiltonian:

$$
H_{c} = U(c) + m[\phi(k) - c - (n + \delta)k]
$$

The derivative conditions are as follow:

\begin{equation} \label{eq1}
 \frac{\partial H_{c}}{\partial c} = U^{\prime}(c) - m = 0
\end{equation}

\begin{equation} \label{eq2}
  \dot{k} = \frac{\partial H_{c}}{\partial m} = \phi(k) - c - (n + \delta)k
\end{equation}

\begin{equation*}
\dot{m} &= -\frac{\partial H_{c}}{\partial k} + rm = -m[\phi^{\prime}(k) - (n + \delta)] + rm \\
\end{equation*}

\begin{equation} \label{eq3}
\Leftrightarrow  \dot{m} &= -m[\phi^{\prime}(k) - (n + \delta + r)]
\end{equation}
\end{frame}

\begin{frame}[label={sec:orgc90649a}]{Constructing the Phase Diagram}
From (\ref{eq1}) and (\ref{eq3}) we have:

\begin{equation} \label{eq4}
  \dot{c} = - \frac{U^{\prime}(c)}{U^{\prime\prime}(c)}\left[\phi^{\prime}(k) - (n + \delta + r)\right]
\end{equation}

Along with (\ref{eq2}):

$$
\dot{k} = \frac{\partial H_{c}}{\partial m} = \phi(k) - c - (n + \delta)k
$$

We calculate the value where \(\dot{c} = 0\) and \(\dot{k} = 0\):

\begin{align}
  \begin{cases}
    (\ref{eq2}) &\Rightarrow \dot{k}=0 \Leftrightarrow c = \phi(k) - (n + \delta)k \\
    (\ref{eq4}) &\Rightarrow \dot{c}=0 \Leftrightarrow \phi^{\prime}(k) = n + \delta + r
  \end{cases}
\end{align}
\end{frame}

\begin{frame}[label={sec:org24e138e}]{Phase diagram analysis}
The \(c-k\) phase diagram (Fig. 2.3 \cite{romer19_advan}):

\begin{center}
\includegraphics[width=.9\linewidth]{/home/hieuphay/Dropbox/Notes/.attach/9d/f184e2-538c-40bc-94f4-590d9dfde5ac/_20200904_073411screenshot.png}
\end{center}
\end{frame}


\begin{frame}[allowframebreaks]{References}
\printbibliography
\end{frame}
\end{document}
